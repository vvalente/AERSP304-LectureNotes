% basic style definitions for lecture notes in text mode.
% based on material from J. W. Langelaan, August 16, 2010

% Vitor T. Valente, August 2024
\DocumentMetadata{
  tagging = on,
  lang = en,
  pdfstandard = ua-2,
  pdfstandard = a-4f, % optional archival standard
  tagging-setup = {math/setup={mathml-AF,mathml-SE},
                   extra-modules={verbatim-mo},
                   table/header-rows=1}}

\documentclass[times,12pt,letterpaper]{article}

\usepackage{latexsym}
\usepackage{fancyhdr}
\usepackage{amsmath, amsthm}
\usepackage{amsfonts}
\usepackage{amssymb}
\usepackage{amsxtra}
\usepackage{graphicx}
\usepackage{epstopdf}
 \usepackage{subfigure}			% subcaptions for subfigures
 \usepackage{subfigmat}			% matrices of similar subfigures, aka small mulitples
\usepackage{times}
\usepackage[small]{caption}
\usepackage{wrapfig}			% wrap figures/tables in text (i.e., Di Vinci style)
\usepackage{color}
\usepackage{ifthen}
\usepackage{float}
\usepackage{multicol}
\usepackage[normalem]{ulem}
\usepackage{titling}
\usepackage{url}
\usepackage{hyperref}


\usepackage{tikz} % For block diagrams
\usetikzlibrary{positioning}

\usepackage{times}
\usepackage{unicode-math}

\newcommand{\myheading}[1]
	{\noindent \textbf{#1}}

\newcommand{\mat}[1]  % define command for bold upright matrix notation (also gives bold greek)
    {\mbox{\boldmath$\mathrm{#1}$}}

\newcommand{\ddt}[1] {\frac{d{#1}}{dt}}
\newcommand{\deldel}[2]{\frac{\delta {#1}}{\delta {#2}}}
\newcommand{\DEG}[1] {\mbox{${#1}^{\circ}$}}
\newcommand{\LAPL}[1]{\mathcal{L}\left\{{#1}\right\}}
\newcommand{\iLAPL}[1]{\mathcal{L}^{-1}\left\{{#1}\right\}}

\pagestyle{fancy}

\newcommand{\course}[1]{\lhead{\textbf{ \footnotesize{#1}}}}
\chead{}
\newcommand{\lecture}[1]{\rhead{\textbf{\footnotesize{#1--\thepage}}}}
\newcommand{\semester}[1]{\lfoot{\textbf{\footnotesize{#1}}}}
\cfoot{}
\rfoot{\textbf{}}
\renewcommand{\headrulewidth}{0pt}
\renewcommand{\footrulewidth}{0pt}

\newcommand{\printnotes}{\printnotestrue}

% Prevent word splitting (hyphenation) and penalize it if it occurs
\hyphenpenalty=10000
\exhyphenpenalty=10000
\sloppy

% define an environment that prints out notes to self in red if \printnotes is true, in white (on white paper,
% hence invisible) otherwise. Printing notes in white will leave space for student notes...
\newif\ifprintnotes
\newenvironment{mynotes}
{
	\ifprintnotes
		\color{red}
	\else
		\color{white}
	\fi
}
{
	\color{black}
}

\definecolor{codegreen}{rgb}{0,0.6,0}
\definecolor{codegray}{rgb}{0.5,0.5,0.5}
\definecolor{codepurple}{rgb}{0.58,0,0.82}
\definecolor{backcolorPython}{rgb}{0.90,0.95,0.92}
\definecolor{backcolorMatlab}{rgb}{0.95,0.90,0.92}
\definecolor{backcolorCpp}{rgb}{0.95,0.95,0.90}
