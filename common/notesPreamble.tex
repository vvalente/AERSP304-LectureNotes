% Basic style definitions for lecture notes in preamble mode.
% Vitor Valente, August 2024
\usepackage{multicol}
\usepackage[normalem]{ulem}
\usepackage{tikz} % For block diagrams
\usetikzlibrary{positioning}
\usepackage{titling}
\usepackage{url}
\usepackage{float}
\usepackage[margin=1in]{geometry} % Adjust the margin as needed
\usepackage[pdfencoding=auto]{hyperref}
\usepackage[dvipsnames]{xcolor}
\usepackage{times}
\usepackage{unicode-math}

% for accessibility
\hypersetup{
pdftitle={AERSP 304 Material},
pdfauthor={V.T. Valente},
pdfkeywords={AERSP304},
pdfdisplaydoctitle=true
}

% suppress page numbers
\pagenumbering{gobble}

% define an environment that prints out notes to self in red if \printnotes is true, in white (on white paper,
% hence invisible) otherwise. Printing notes in white will leave space for student notes...
\usepackage{environ}

\newif\ifprintnotes
\printnotesfalse % set true to show notes

\newsavebox{\mynotesbox}

\NewEnviron{mynotes}{%
  % 1) Measure content
  \sbox{\mynotesbox}{%
    \begin{minipage}{\linewidth}
			\color{BrickRed}%
      \BODY
    \end{minipage}
  }%
  % 2) Output or reserve space
  \ifprintnotes
    \par\noindent\usebox{\mynotesbox}\par%
  \else
    \par\noindent\phantom{\usebox{\mynotesbox}}\par
  \fi
}

\NewEnviron{mynotesfigures}{%
  \sbox{\mynotesbox}{%
    \begin{minipage}{\linewidth}
      \centering
      \BODY
    \end{minipage}
  }%
  \ifprintnotes
    \par\noindent\usebox{\mynotesbox}\par%
  \else
    \par\noindent\phantom{\usebox{\mynotesbox}}\par
  \fi
}


\newcommand{\myheading}[1]
	{\noindent \textbf{#1}}

\newcommand{\mat}[1]  % define command for bold upright matrix notation (also gives bold greek)
    {\mbox{\boldmath$\mathrm{#1}$}}

\newcommand{\ddt}[1] {\frac{d{#1}}{dt}}
\newcommand{\deldel}[2]{\frac{\delta {#1}}{\delta {#2}}}
\newcommand{\DEG}[1] {\mbox{${#1}^{\circ}$}}
\newcommand{\LAPL}[1]{\mathcal{L}\left\{{#1}\right\}}
\newcommand{\iLAPL}[1]{\mathcal{L}^{-1}\left\{{#1}\right\}}

% Prevent word splitting (hyphenation) and penalize it if it occurs
\hyphenpenalty=10000
\exhyphenpenalty=10000
\sloppy