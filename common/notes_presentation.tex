% basic style definitions for lecture notes in presentation mode.
% based on material from J. W. Langelaan, August 16, 2010

% Vitor Valente, August 2024

\documentclass[times,12pt]{beamer}

\setbeamercolor{background canvas}{bg=white} % Background is set to white
\setbeamercolor{normal text}{fg=black}
\setbeamercolor{alerted text}{fg=black}
\setbeamercolor{example text}{fg=black}
\setbeamercolor{structure}{fg=black} % Section headers, etc.

\setbeamerfont{section title}{size=\small}
\setbeamerfont{frametitle}{size=\large}
\setbeamerfont{normal text}{size=\small}

\setbeamertemplate{navigation symbols}{}

\setbeamertemplate{itemize subitem}[circle]

\setbeamerfont{normal text}{size=\footnotesize}
\setbeamerfont{itemize item}{size=\footnotesize}
\setbeamerfont{itemize text}{size=\footnotesize}

\usepackage{graphicx}
\usepackage{amsmath}  % For mathematical symbols
\usepackage{graphicx} % For including images
\usepackage{hyperref} % For clickable links
\usepackage{subcaption}
\usepackage{tikz} % For block diagrams
\usetikzlibrary{positioning}

\usepackage{ragged2e}
\apptocmd{\frame}{}{\justifying}{} % Allow optional arguments after frame.

\usepackage{xcolor}

% define an environment that prints out notes to self in red if \printnotes is true, in white (on white paper,
% hence invisible) otherwise. Printing notes in white will leave space for student notes...
\newif\ifprintnotes
\newenvironment{mynotes}
{
	\ifprintnotes
		\color{red}
	\else
		\color{white}
	\fi
}
{
	\color{black}
}
