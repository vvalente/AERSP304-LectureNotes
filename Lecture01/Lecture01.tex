\DocumentMetadata{
 lang=en,
 pdfversion=2.0,
 pdfstandard=ua-2,
 %pdfstandard=a-4f, % needs to be remove so Canvas Ally accepts title
 tagging=on,
 tagging-setup={math/setup=mathml-SE},
 debug={xmp-export}
}

% !TEX root = Lecture01.tex
\documentclass[frame-title-arg,aspect-ratio=4:3]{ltx-talk}

% Basic style definitions for lecture notes in presentation mode.
% Vitor Valente, August 2024
\usepackage{graphicx}
\usepackage{amsmath}  % For mathematical symbols
\usepackage{graphicx} % For including images
\usepackage{hyperref} % For clickable links
\usepackage{tikz} % For block diagrams
\usetikzlibrary{positioning}

\usepackage{ragged2e}
\apptocmd{\frame}{}{\justifying}{} % Allow optional arguments after frame.

\usepackage{multicol}
\usepackage[normalem]{ulem}
\usepackage{url}
\usepackage{float}
\usepackage{fontspec}
\setmainfont{TeX Gyre Termes}
\usepackage{pgfpages}
\usepackage[dvipsnames]{xcolor}
\usepackage{comment}
\usepackage{unicode-math}

% setting up style
\EditInstance{header}{std}{
  print-frame-title = true,
  color = black,
}

\EditInstance{frametitle}{header}{
  font = \bfseries\LARGE,
}

% for accessibility
\hypersetup{
pdftitle={AERSP 304 Material},
pdfauthor={V.T. Valente},
pdfkeywords={AERSP304},
pdfdisplaydoctitle=true
}

% for printing purposes, 2 slides per page with borders
\pgfpagesuselayout{2 on 1}[letterpaper,border shrink=5mm]
\pgfpageslogicalpageoptions{1}{border code=\pgfusepath{stroke}}
\pgfpageslogicalpageoptions{2}{border code=\pgfusepath{stroke}}
\pgfpageslogicalpageoptions{3}{border code=\pgfusepath{stroke}}
\pgfpageslogicalpageoptions{4}{border code=\pgfusepath{stroke}}

% suppress page numbers
\pagenumbering{gobble}
% black items
% \AtBeginEnvironment{itemize}{\color{black}}
% \AtBeginEnvironment{enumerate}{\color{black}}

% define an environment that prints out notes to self in red if \printnotes is true, in white (on white paper,
% hence invisible) otherwise. Printing notes in white will leave space for student notes...
\usepackage{environ}

\newif\ifprintnotes
\printnotesfalse % set true to show notes

\newsavebox{\mynotesbox}

\NewEnviron{mynotes}{%
  % 1) Measure content
  \sbox{\mynotesbox}{%
    \begin{minipage}{\linewidth}
			\color{blue}%
      \BODY
    \end{minipage}
  }%
  % 2) Output or reserve space
  \ifprintnotes
    \usebox{\mynotesbox}%
  \else
    \par\noindent\phantom{\usebox{\mynotesbox}}\par
  \fi
}

\NewEnviron{mynotesfigures}{%
  \sbox{\mynotesbox}{%
    \begin{minipage}{\linewidth}
      \centering
      \BODY
    \end{minipage}
  }%
  \ifprintnotes
    \usebox{\mynotesbox}%
  \else
    \par\noindent\phantom{\usebox{\mynotesbox}}\par
  \fi
}


\printnotestrue % set to true to print notes in red,
% \printnotesfalse % set to false to hide notes (white on white paper)

% ======= INIT =========
\begin{document}

\section{Introduction}
\begin{frame}{}
    \begin{center}
      \huge Introduction to Dynamics and Control\\
      \normalsize AERSP 304 - Dynamics and Control of Aerospace Systems\\
      \vspace{1cm}
      \normalsize V.T. Valente\\
      \vspace{0.5cm}
      \small Penn State University\\
      \vspace{0.5cm}
      \normalsize January 12, 2026
    \end{center}
\end{frame}

\begin{frame}{Goals for Today}
    \begin{itemize}
        \item Review Syllabus for SP26
        \item Introduction of Dynamics and Control
    \end{itemize}
\end{frame}

\section{Course Overview}
\begin{frame}{Course Information}
  \begin{center}
      \large AERSP304 Dynamics and Control of Aerospace Systems \\
      \large Section 002
  \end{center}
  \vspace{0.5cm}
  \textbf{Instructor:} Prof. Vitor Valente \\
  \textbf{Email:} vitor.valente@psu.edu \\
  \textbf{Office:} 473N ECoRE \\
  \textbf{Office Hours:} TR 10:30 AM - 11:30 AM or by appointment \\
  \textbf{Textbook:} N. Nise, “Control System Engineering,” John Wiley \& Sons, 2019 \\
  \vspace{1cm}
  \textbf{Lectures:} \\
  \begin{itemize}
      \setlength{\itemsep}{0.0cm}
      \setlength{\parsep}{0cm}
      \item Location: 102 ECoRE Building \\
      \item Meeting Time: MWF 10:10 AM – 11:00 AM \\
  \end{itemize}
\end{frame}

\begin{frame}{Goals for the Course}
    \begin{itemize}
        \item Bridge the gap between mathematical theory and application to real engineering systems
        \item Develop systematic methods for modeling complex dynamical systems with single and multiple degrees of freedom
        \item Learn how to approximate complex engineering system with linear ordinary differential equations
        \item Integrate concepts from dynamics, linear algebra, differential equations, and control theory
    \end{itemize}
\end{frame}

\begin{frame}{}

    \tagstructbegin{tag=Sect}%required to ensure correct reading order of contents if the table has any caption
    \begin{figure}[H]
        \centering
        \includegraphics[width=0.75\textwidth,alt={SpaceX}]{images/spacex.png}
    \end{figure}
    \tagstructend

    {
    \vspace{0.25cm}
    \fontsize{4}{6}\selectfont
    \hspace{2.0cm} https://www.marketwatch.com/story/space-stocks-rise-after-spacex-captures-booster-in-fifth-starship-flight-test-4946087d
    }
\end{frame}

% \begin{frame}{Another Example}
%     \begin{itemize}
%         \item Humans and other animals constantly perform feedback control in normal activity
%         \item Humans are “inverted pendulums”: we cannot stand up without subconsciously performing feedback control
%         \begin{itemize}
%             \item Sensor: Inner ear (vestibular system) and vision
%             \item Actuators: Muscles
%             \item Controller: Cerebellum (brain part that controls coordination and balance)
%         \end{itemize}
%     \end{itemize}
% \end{frame}

\begin{frame}{Overall Course Structure}
    \begin{itemize}
        \item Classes meet in-person on scheduled times
        \item Lectures may be conducted remotely (recorded and made available)
        \item All lectures are recorded via Zoom and available on CANVAS Media Gallery
        \item Annotated lecture slides and supplementary materials are provided on CANVAS
    \end{itemize}
\end{frame}

\begin{frame}{Overall Course Structure Continued}
    The course follows a four-stage progression:
    \begin{enumerate}
        \item Modeling
        \item Stability and Performance Analysis
        \item Analysis and Control Design in Time Domain
        \item Control Design in Frequency Domain
    \end{enumerate}
\end{frame}

\subsection{Assignments}
\begin{frame}{Assignments: Summary}

    \tagstructbegin{tag=Sect}%required to ensure correct reading order of contents if the table has any caption
    \begin{figure}[H]
        \centering
        \includegraphics[width=0.9\textwidth,alt={Assignments}]{images/assignments.png}
    \end{figure}
    \tagstructend

    \vspace{1cm}
    \begin{itemize}
        \item GradeScope
    \end{itemize}
\end{frame}

\begin{frame}{Assignments - Quizzes}
    \textbf{Short Quizzes:}
    \begin{itemize}
        \setlength{\itemsep}{0.0cm}
        \setlength{\parsep}{0cm}
        \item Every Thursday, closed book
        \item Conceptual understanding rather than lengthy derivations
        \item No make-up quizzes; lowest score is dropped
    \end{itemize}
    \vspace{0.5cm}
    \textbf{In-Class Homework Quizzes:}
    \begin{itemize}
        \setlength{\itemsep}{0.0cm}
        \setlength{\parsep}{0cm}
        \item See schedule in Syllabus, closed-book
        \item Not collected or graded
        \item One problem from the HW set
        \item No make-up quizzes; lowest score is dropped
    \end{itemize}
    \vspace{0.5cm}
    Solutions to HW will be posted after in-class quiz
\end{frame}

\begin{frame}{Assignments - Exams}
    \textbf{Exams:}
    \begin{itemize}
        \setlength{\itemsep}{0.0cm}
        \setlength{\parsep}{0cm}
        \item 2 Midterms (Feb 13 and Mar 31) and 1 Final Exam (TBD)
        \item Exam problems are closely aligned with homework problems
        \item No practice exams will be provided
        \item One paper sheet with handwritten notes allowed per exam
        \begin{itemize}
            \item No problem solutions
        \end{itemize}
        \item Make-up exams are available for legitimate reasons only and need to be scheduled in advance
        \item Highest score gets 2\% additional weight on final grade
    \end{itemize}
\end{frame}

\begin{frame}{Assignments - Projects}
    \small
    Students select one of two project tracks, working in groups of up to three:
    \begin{itemize}
        \setlength{\itemsep}{0.0cm}
        \setlength{\parsep}{0cm}
        \item \textbf{Simulation}: Three MATLAB projects aligned with modeling, analysis, and control modules.
        \item \textbf{Funduino}: One MATLAB project plus two hardware-focused projects involving system identification and control.
    \end{itemize}

    \vspace{0.15cm}

    \tagstructbegin{tag=Sect}%required to ensure correct reading order of contents if the table has any caption
    \begin{figure}[H]
        \centering
        \includegraphics[width=0.45\textwidth,alt={Funduino Project}]{images/funduino.png}
    \end{figure}
    \tagstructend

    {
      \vspace{-0.2cm}
      \hspace{3.5cm} \fontsize{4}{6}\selectfont{Students Funduino Project presentation.}
    }

\end{frame}

\begin{frame}{Funduino Option}
    \small
    \begin{itemize}
        \item Use of Arduino-based hardware for hands on experience with real-time control systems
        \item Project includes:
        \begin{itemize}
            \item Modeling actuator using experimental data
            \item Controller design and performance evaluation
        \end{itemize}
        \item Emphasis on practical skills in data acquisition, signal processing, and real-time control
        \item Demonstrations, midterm report and final report
        \item Student can earn 2\% bonus on final grade
    \end{itemize}
    Students are advised to commit early or switch tracks before deadline (due date for 2nd Matlab Project).
\end{frame}

% \begin{frame}{How to Succeed in this Course?}
%     \begin{itemize}
%         \item Keep up with the class schedule
%         \item Do not miss more than one Quiz
%         \item Do not miss more than one HW Quiz
%         \item Solve the homework problems
%         \item Take advantage of opportunities to communicate with instructor and classmates
%     \end{itemize}
% \end{frame}

\begin{frame}
    \centering
    \huge Introduction to Dynamics and Control
\end{frame}

\section{Dynamics and Control}
\begin{frame}{Dynamics and Control}

    \textbf{What is a model?}\\
    \begin{mynotes}
     It is a set of dynamic equations that describe the behavior of a (process/plant) system. \par
     \ifprintnotes
      \begin{itemize}
          \item Physics based models: derived from first principles (e.g., Newton’s laws, conservation of mass/energy)
          \item Data driven models: derived from experimental data (e.g., system identification)
      \end{itemize}
      \else
        \vspace{2.0cm}
      \fi
    \end{mynotes}

    \textbf{What is feedback control?}\\
    \begin{mynotes}
      Feedback control is the use of measurements of system output to influence system inputs in order to achieve desired performance. \par
      \vspace{0.5cm}
      \ifprintnotes

      \tagstructbegin{tag=Sect}%required to ensure correct reading order of contents if the table has any caption
      \begin{figure}[H]
          \centering
          \includegraphics[width=0.65\textwidth,alt={Example Temperature Control}]{images/thermostat.png}
      \end{figure}
      \tagstructend
      \else
      \vspace{2.0cm}
      \fi
      \vspace{-1.0cm}
    \end{mynotes}
\end{frame}

\begin{frame}{Dynamics of Mechanical Systems}
    Elements: Mass, Inertia, Springs, Dampers, Forces, Torques \\
    \vspace{0.25cm}
    \textbf{Translational Motion:}
    \begin{itemize}
        \item Newton’s Second Law:
        \[
        F = ma
        \]
        \item E.g.: Cruise Control
    \end{itemize}
    \vspace{0.25cm}
    \textbf{Rotational Motion:}
    \begin{itemize}
        \item Newton’s Second Law:
        \[
        M = I\alpha
        \]
        \item E.g.: Satellite Attitude Control, Read/Write Disk Drive, Pendulum
    \end{itemize}
    \vspace{0.25cm}
    \textbf{Combined Rotation and Translation:}
    \begin{itemize}
        \item E.g.: Pendulum on a Moving Cart
    \end{itemize}
\end{frame}

\begin{frame}{Dynamics of Electric Circuits}
    Circuit elements: Resistor, Capacitor, Inductor, Voltage and Current Sources, Operational Amplifiers \\
    \vspace{0.25cm}
    \textbf{Circuit Analysis:}
    \begin{itemize}
        \item Kirchhoff’s Current and Voltage Laws
        \item E.g.: RLC, Integrator
    \end{itemize}
    \vspace{0.25cm}
    Electric systems are often combined with mechanical systems \\
    \begin{itemize}
        \item E.g.: DC Motor, Loudspeaker
    \end{itemize}
\end{frame}

\begin{frame}{Dynamics of Other Systems}
    \textbf{Thermal Systems:}
    \begin{itemize}
        \item Conservation of Energy
        \item E.g.: Temperature Control of a Heated Room
    \end{itemize}
    \textbf{Hydraulic Systems:}
    \begin{itemize}
        \item Continuity
        \item E.g.: Water Level on a Tank, Hydraulic Actuator
    \end{itemize}

    \begin{mynotes}

    \ifprintnotes

      \tagstructbegin{tag=Sect}%required to ensure correct reading order of contents if the table has any caption
      \begin{figure}[H]
          \centering
          \includegraphics[width=0.45\textwidth,alt={Example Tank}]{images/tank.png}
      \end{figure}
      \tagstructend
    \else
        \vspace{2.0cm}
    \fi

      % add equation of the tank height dynamics
      \[
      \frac{dh(t)}{dt} = \frac{1}{\rho A} (w_{in} - w_{out} ), \quad h = \frac{m}{\rho A}
      \]
    \end{mynotes}
    \vspace{-0.5cm}
\end{frame}

\begin{frame}{Force-Voltage Analogy for Dynamic Systems}
    \vspace{0pt}
    Mathematical equations for different physical systems can be made equivalent using analogies \\
    \vspace{0.5cm}

    \tagstructbegin{tag=Sect}%required to ensure correct reading order of contents if the table has any caption
    \begin{figure}[H]
        \centering
        \includegraphics[width=0.25\textwidth,alt={Mechanical System}]{images/mechanical.png}
        \hspace{1cm}
        \includegraphics[width=0.5\textwidth,alt={Electrical System}]{images/electrical.png}
    \end{figure}
    \tagstructend

    \begin{mynotes}
      % add equation of the mechanical system
      \[
      M\frac{d^2x(t)}{dt^2} + B\frac{dx(t)}{dt} + Kx(t) = F(t)
      \]

      % add equation of the electrical system
      \[
      L\frac{d^2q(t)}{dt^2} + R\frac{dq(t)}{dt} + \frac{1}{C}q(t) = V(t)
      \]

      Force <--> Voltage, Mass <--> Inductance, Friction <--> Resistance, Spring Constant <--> Reciprocal of Capacitance, Displacement <--> Charge, Velocity <--> Current
    \end{mynotes}
    \vspace{-0.5cm}
\end{frame}

% Translational Mechanical System	Electrical System
% Force(F)	Voltage(V)
% Mass(M)	Inductance(L)
% Frictional Coefficient(B)	Resistance(R)
% Spring Constant(K)	Reciprocal of Capacitance (1/C)
% Displacement(x)	Charge(q)
% Velocity(v)	Current(i)

\end{document}
