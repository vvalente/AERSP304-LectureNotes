% basic style definitions for lecture notes in presentation mode.
% based on material from J. W. Langelaan, August 16, 2010

% Vitor Valente, August 2024

\documentclass[times,12pt]{beamer}

\setbeamercolor{background canvas}{bg=white} % Background is set to white
\setbeamercolor{normal text}{fg=black}
\setbeamercolor{alerted text}{fg=black}
\setbeamercolor{example text}{fg=black}
\setbeamercolor{structure}{fg=black} % Section headers, etc.

\setbeamerfont{section title}{size=\small}
\setbeamerfont{frametitle}{size=\large}
\setbeamerfont{normal text}{size=\small}

\setbeamertemplate{navigation symbols}{}

\setbeamertemplate{itemize subitem}[circle]

\setbeamerfont{normal text}{size=\footnotesize}
\setbeamerfont{itemize item}{size=\footnotesize}
\setbeamerfont{itemize text}{size=\footnotesize}

\usepackage{graphicx}
\usepackage{amsmath}  % For mathematical symbols
\usepackage{graphicx} % For including images
\usepackage{hyperref} % For clickable links
\usepackage{subcaption}
\usepackage{tikz} % For block diagrams
\usetikzlibrary{positioning}

\usepackage{ragged2e}
\apptocmd{\frame}{}{\justifying}{} % Allow optional arguments after frame.

\usepackage{xcolor}

% define an environment that prints out notes to self in red if \printnotes is true, in white (on white paper,
% hence invisible) otherwise. Printing notes in white will leave space for student notes...
\newif\ifprintnotes
\newenvironment{mynotes}
{
	\ifprintnotes
		\color{red}
	\else
		\color{white}
	\fi
}
{
	\color{black}
}

% !TEX root = Lecture01.tex
\usepackage{pgfpages}
\usepackage[absolute,overlay]{textpos}

\usetikzlibrary{calc}
\usetikzlibrary{arrows.meta, positioning}

% For printing purposes, 2 slides per page with borders
\pgfpagesuselayout{2 on 1}[letterpaper,border shrink=5mm]
\pgfpageslogicalpageoptions{1}{border code=\pgfusepath{stroke}}
\pgfpageslogicalpageoptions{2}{border code=\pgfusepath{stroke}}
\pgfpageslogicalpageoptions{3}{border code=\pgfusepath{stroke}}
\pgfpageslogicalpageoptions{4}{border code=\pgfusepath{stroke}}

% Title Page Information
\title[AERSP304]{Introduction to Dynamics and Control}
\subtitle{AERSP 304 - Dynamics and Control of \\ Aerospace Systems}
\author[V. T. Valente]{V.T. Valente}
\institute[Penn State University]{Penn State University}
\date{January 12, 2026}

\setlength{\itemsep}{0.0cm}
\setlength{\parsep}{0cm}

\hypersetup{
  pdftitle={Introduction to Dynamics and Control - AERSP304},
  pdfauthor={V. T. Valente},
  pdfsubject={Dynamics and Control of Aerospace Systems},
  pdfkeywords={Dynamics, Control, Aerospace Systems, AERSP304},
  pdfdisplaydoctitle=true
}

\begin{document}
{
%\setbeamertemplate{footline}[text line]{\parbox{\linewidth}{\vspace*{-8pt}Some material in this presentation has been adapted from slides developed by Dr. Joe Horn.}}
\frame{\titlepage}
}

\small

% Course Information Slide
\begin{frame}{Course Information}
    \begin{center}
        \textbf{AERSP304 Dynamics and Control of Aerospace Systems} \\
        Section 002
    \end{center}
    \vspace{0.5cm}
    \textbf{Instructor:} Prof. Vitor Valente \\
    \textbf{Email:} vitor.valente@psu.edu \\
    \textbf{Office:} 473N ECoRE \\
    \textbf{Office Hours:} TR 10:30 AM - 11:30 AM or by appointment \\
    \textbf{Textbook:} N. Nise, “Control System Engineering,” John Wiley \& Sons, 2019 \\
    \vspace{1cm}
    \textbf{Lectures:} \\
    \begin{itemize}
        \setlength{\itemsep}{0.0cm}
        \setlength{\parsep}{0cm}
        \item Location: 102 ECoRE Building \\
        \item Meeting Time: MWF 10:10 AM – 11:00 AM \\
    \end{itemize}
\end{frame}

% \begin{frame}
%     \begin{figure}[h!]
%         \centering
%         \includegraphics[width=1.0\textwidth]{images/defense_inc_pes_opt_thumbnail.png}
%         \end{figure}
%     \href{run:/Applications/VLC.app Lecture01.mp4}{Video}
% \end{frame}

\begin{frame}{Goals for Today}
    \begin{itemize}
        \item Review Syllabus for SP26
        \item Introduction of Dynamics and Control
    \end{itemize}
\end{frame}

% Goals Slide
\begin{frame}{Goals for the Course}
    \begin{itemize}
        \item Bridge the gap between mathematical theory and application to real engineering systems
        \item Develop systematic methods for modeling complex dynamical systems with single and multiple degrees of freedom
        \item Learn how to approximate complex engineering system with linear ordinary differential equations
        \item Integrate concepts from dynamics, linear algebra, differential equations, and control theory
    \end{itemize}
\end{frame}

\begin{frame}{Goals for the Course}

    \tagstructbegin{tag=Sect}%required to ensure correct reading order of contents if the table has any caption
    \begin{figure}[H]
        \centering
        \includegraphics[width=0.75\textwidth,alt={SpaceX}]{images/spacex.png}
    \end{figure}
    \tagstructend

    {
    \vspace{-0.25cm}
    \fontsize{4}{6}\selectfont
    \hspace{1.8cm} https://www.marketwatch.com/story/space-stocks-rise-after-spacex-captures-booster-in-fifth-starship-flight-test-4946087d
    }
\end{frame}

% \begin{frame}{Another Example}
%     \begin{itemize}
%         \item Humans and other animals constantly perform feedback control in normal activity
%         \item Humans are “inverted pendulums”: we cannot stand up without subconsciously performing feedback control
%         \begin{itemize}
%             \item Sensor: Inner ear (vestibular system) and vision
%             \item Actuators: Muscles
%             \item Controller: Cerebellum (brain part that controls coordination and balance)
%         \end{itemize}
%     \end{itemize}
% \end{frame}

\begin{frame}{Overall Course Structure}
    \begin{itemize}
        \item Classes meet in-person on scheduled times
        \item Lectures may be conducted remotely (recorded and made available)
        \item All lectures are recorded via Zoom and available on CANVAS Media Gallery
        \item Annotated lecture slides and supplementary materials are provided on CANVAS
    \end{itemize}
\end{frame}

\begin{frame}{Overall Course Structure Continued}
    The course follows a four-stage progression:
    \begin{enumerate}
        \item Modeling
        \item Stability and Performance Analysis
        \item Analysis and Control Design in Time Domain
        \item Control Design in Frequency Domain
    \end{enumerate}
\end{frame}

\begin{frame}{Assignments: Summary}

    \tagstructbegin{tag=Sect}%required to ensure correct reading order of contents if the table has any caption
    \begin{figure}[H]
        \centering
        \includegraphics[width=1.0\textwidth]{images/assignments.png}
    \end{figure}
    \tagstructend

    \begin{itemize}
        \item GradeScope
    \end{itemize}
\end{frame}

\begin{frame}{Assignments - Quizzes}
    \textbf{Short Quizzes:}
    \begin{itemize}
        \setlength{\itemsep}{0.0cm}
        \setlength{\parsep}{0cm}
        \item Every Thursday, closed book
        \item Conceptual understanding rather than lengthy derivations
        \item No make-up quizzes; lowest score is dropped
    \end{itemize}
    \textbf{In-Class Homework Quizzes:}
    \begin{itemize}
        \setlength{\itemsep}{0.0cm}
        \setlength{\parsep}{0cm}
        \item See schedule in Syllabus, closed-book
        \item Not collected or graded
        \item One problem from the HW set
        \item No make-up quizzes; lowest score is dropped
    \end{itemize}
    Solutions to HW will be posted after in-class quiz
\end{frame}

\begin{frame}{Assignments - Exams}
    \textbf{Exams:}
    \begin{itemize}
        \setlength{\itemsep}{0.0cm}
        \setlength{\parsep}{0cm}
        \item 2 Midterms (Feb 13 and Mar 31) and 1 Final Exam (TBD)
        \item Exam problems are closely aligned with homework problems
        \item No practice exams will be provided
        \item One paper sheet with handwritten notes allowed per exam
        \begin{itemize}
            \item No problem solutions
        \end{itemize}
        \item Make-up exams are available for legitimate reasons only and need to be scheduled in advance
        \item Highest score gets 2\% additional weight on final grade
    \end{itemize}
\end{frame}

\begin{frame}{Assignments - Projects}
    \small
    Students select one of two project tracks, working in groups of up to three:
    \begin{itemize}
        \setlength{\itemsep}{0.0cm}
        \setlength{\parsep}{0cm}
        \item \textbf{Simulation}: Three MATLAB projects aligned with modeling, analysis, and control modules.
        \item \textbf{Funduino}: One MATLAB project plus two hardware-focused projects involving system identification and control.
    \end{itemize}
    \vspace{-0.5cm}

    \tagstructbegin{tag=Sect}%required to ensure correct reading order of contents if the table has any caption
    \begin{figure}[h!]
        \centering
        \includegraphics[width=0.35\textwidth]{images/funduino.png}
    \end{figure}
    \tagstructend
    {
    \vspace{-0.5cm}
    \fontsize{4}{6}\selectfont
    {\hspace{3.5cm}Students Funduino Project presentation.}
    }
\end{frame}

\begin{frame}{Funduino Option}
    \small
    \begin{itemize}
        \item Use of Arduino-based hardware for hands on experience with real-time control systems
        \item Project includes:
        \begin{itemize}
            \item Modeling actuator using experimental data
            \item Controller design and performance evaluation
        \end{itemize}
        \item Emphasis on practical skills in data acquisition, signal processing, and real-time control
        \item Midterm report and final report
        \item Demonstrations
        \item Student can earn 2\% bonus on final grade
    \end{itemize}
    Students are advised to commit early or switch tracks before deadline (due date for 2nd Matlab Project).
\end{frame}

% \begin{frame}{How to Succeed in this Course?}
%     \begin{itemize}
%         \item Keep up with the class schedule
%         \item Do not miss more than one Quiz
%         \item Do not miss more than one HW Quiz
%         \item Solve the homework problems
%         \item Take advantage of opportunities to communicate with instructor and classmates
%     \end{itemize}
% \end{frame}

\begin{frame}
    \centering
    \Huge Introduction to Dynamics and Control
\end{frame}

\begin{frame}[t]{Dynamics and Control}
    Model: set of dynamic equations that describe the behavior of a (process) system \\
    \begin{itemize}
        \setlength{\itemsep}{0.0cm}
        \setlength{\parsep}{0cm}
        \item Physics based models: derived from first principles (e.g., Newton’s laws, conservation of mass/energy)
        \item Data driven models: derived from experimental data (e.g., system identification)
    \end{itemize}
    Feedback control: use of measurements of system output to influence system input \\
    \vspace{0.25cm}
    Often including assumptions to simplify analysis and design
\end{frame}

\begin{frame}[t]{Dynamics of Mechanical Systems}
    Elements: Mass, Inertia, Springs, Dampers, Forces, Torques \\
    \vspace{0.5cm}
    Translational Motion:
    \begin{itemize}
        \item Newton’s Second Law: $F = ma$
        \item E.g.: Cruise Control
    \end{itemize}
    \vspace{0.25cm}
    Rotational Motion:
    \begin{itemize}
        \item Newton’s Second Law: $M = I\alpha$
        \item E.g.: Satellite Attitude Control, Read/Write Disk Drive, Pendulum
    \end{itemize}
    Combined Rotation and Translation:
    \begin{itemize}
        \item E.g.: Pendulum on a Moving Cart
    \end{itemize}
\end{frame}

\begin{frame}[t]{Dynamics of Electric Circuits}
    Circuit elements: Resistor, Capacitor, Inductor, Voltage and Current Sources, Operational Amplifiers \\
    \vspace{0.5cm}
    \textbf{Circuit Analysis:}
    \begin{itemize}
        \item Governing equations: Kirchhoff’s Current and Voltage Laws
        \item E.g.: RLC, Integrator
    \end{itemize}
    \vspace{1cm}
    Electric systems are often combined with mechanical systems \\
    \begin{itemize}
        \item E.g.: DC Motor, Loudspeaker
    \end{itemize}
\end{frame}

\begin{frame}[t]{Dynamics of Other Systems}
    \textbf{Thermal Systems:}
    \begin{itemize}
        \item Governing equations: Conservation of Energy
        \item E.g.: Temperature Control of a Heated Room
    \end{itemize}
    \textbf{Hydraulic Systems:}
    \begin{itemize}
        \item Governing equations: Continuity
        \item E.g.: Water Level on a Tank, Hydraulic Actuator
    \end{itemize}
\end{frame}

\begin{frame}[t]{Force-Voltage Analogy for Dynamic Systems}
    Mathematical equations for different physical systems can be made equivalent using analogies \\
    \vspace{0.5cm}

    \tagstructbegin{tag=Sect}%required to ensure correct reading order of contents if the table has any caption
    \begin{figure}[H]
        \centering
        \includegraphics[width=0.25\textwidth]{images/mechanical.png}
        \hspace{1cm}
        \includegraphics[width=0.5\textwidth]{images/electrical.png}
    \end{figure}
    \tagstructend
\end{frame}

% Translational Mechanical System	Electrical System
% Force(F)	Voltage(V)
% Mass(M)	Inductance(L)
% Frictional Coefficient(B)	Resistance(R)
% Spring Constant(K)	Reciprocal of Capacitance (1c)
% Displacement(x)	Charge(q)
% Velocity(v)	Current(i)

\end{document}
