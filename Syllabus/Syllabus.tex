\input{../common/notes_preamble.tex}
% !TEX root = Syllabus.tex

\usepackage{multicol}
\usepackage[normalem]{ulem}
\usepackage{titling}
\usepackage{url}
\usepackage{hyperref}
\usepackage[margin=1in]{geometry} % Adjust the margin as needed
\pagenumbering{gobble}

\usepackage{titlesec}
\titleformat{\section}
    {\color{black}\normalfont\Large\bfseries}
    {\thesection}{1em}{}
\titlespacing*{\section}{0pt}{0cm}{1em}

\title{AERSP 304 - Dynamics and Control of \\ Aerospace Systems, Section 002 \\ {\fontsize{14pt}{16pt}\selectfont SP26, MWF 10:10AM-11:00AM, ECoRE Bldg 102}}
\author{}
\date{}

\begin{document}
\maketitle
\vspace{-3cm}
% Start here
% Prevent word splitting (hyphenation) and penalize it if it occurs
\hyphenpenalty=10000
\exhyphenpenalty=10000
\sloppy

\begin{multicols}{2}
\subsection*{Instructor Information}
Dr. Vitor T. Valente \\
Office: 473N ECoRE Building \\
Email: vitor.valente@psu.edu \\
Office hours: \\
\indent TR 10:30AM-11:30AM \\
\indent or by appointment

\columnbreak
\subsection*{Teaching Assistant Information}
name \\
Email: email@psu.edu \\
Office hours:  \\
\indent when \\
\indent or by appointment

\end{multicols}

\subsection*{Course Description}
In various courses prior to AERSP 304, you have learned basic physical laws and analysis topics in many distinct fields, such as statics, dynamics, structures, electrical systems, mechanics of materials. In these courses, you were presented with relatively small-scale, simple versions of problems involving each of these topics by itself. The goal of those courses was to learn the fundamental physics (often called ``first principles'') of each separate topic area. Real-world applications usually requires additional mathematical capabilities to solve, analyze, and otherwise understand the behavior of these models.

In AERSP 304, we will focus on learning analytical and computational techniques for solving and/or understanding mathematical models of engineering systems with special emphasis on multiple degrees of freedom systems. Along the way, you will acquire crucial insight into the theory and practice of designing and/or analyzing the kinds of complex engineering systems found throughout aerospace engineering practice.

\subsection*{Objectives}
By the end of this course students will be able to:
\begin{enumerate}
    \setlength{\itemsep}{0pt}
    \item Develop models for single- and multiple degrees-of-freedom systems.
    \item Learn how to approximate complex engineering system with linear ordinary differential equations.
    \item Solve, analyze, and interpret mathematical models of engineering systems, especially models based on linear, time-invariant, ordinary differential equations.
    \item Analyze the stability and response of simple linear feedback control systems.
    \item Get exposed to scientific computing applications to engineering system analysis.
\end{enumerate}

\subsection*{Prerequisites}
Enforced prerequisite at enrollment:
\begin{itemize}
    \setlength{\itemsep}{0pt}
    \item AERSP313
    \item EMCH212
\end{itemize}

\subsection*{Course bibliography}
Recommended textbooks and references:
\begin{itemize}
    \setlength{\itemsep}{0pt}
    \item N. Nise, ``Control System Engineering,'' John Wiley \& Sons, 2019. (\textbf{Textbook})\\
This is a required text for this class and is available at University book store. You can buy $7^{th}$ or $8^{th}$ edition of the book and you may want to check the price at online vendors before purchasing it at bookstore.
    \item K. Ogata, `` Modern Control Engineering,'' Pearson, 2010.
    \item  S. S. Rao,``Mechanical Vibrations,'' Prentice Hall, 2017.
    \item  B. T. Kulakowski, J. F. Gardner, J. L. Shearer, ``Dynamical Modeling and Control of Engineering Systems,'' Cambridge University Press, 2007.
\end{itemize}

\subsection*{Course Structure}

The class will meet in-person on scheduled times but there may be times when lectures will be conducted remotely (a recording will be provided ahead of class time). Every class lecture will be recorded through Zoom and lecture recordings will be available on CANVAS through Media Gallery.

How much and how well you learn is ultimately up to you. You will succeed if you are diligent about keeping up with the class schedule and if you take advantage of opportunities to communicate with me as well as with your fellow students.

Many of us are juggling many obligations and responsibilities between school, work, and family and good time management skills are crucial for academic success. In this respect, I encourage you to follow the advice provided at \url{https://www.worldcampus.psu.edu/about-us/news-and-features/time-management-five-essentials-for-online-learners} to be productive and organized.

\subsubsection*{Detailed Course Content}
The content is subjected to minor changes.
\begin{itemize}
    \item \textbf{Modeling (4 weeks)}
    \begin{itemize}
    \item Review: Modeling of aerospace systems
    \item Time derivative of a vector, Lagrange formulation
    \item Linearization
    \item State space formulation
    \end{itemize}

    \item \textbf{Stability \& Performance  (6 weeks)}
    \begin{itemize}
    \item Stability, Routh criterion
    \item State transition matrix, Forced response
    \item Transfer functions, Block diagrams
    \item Transient response, Steady state errors
    \end{itemize}

    \item  \textbf{Analysis (4 weeks)}
    \begin{itemize}
    \item Root locus
    \item Compensator design
    \end{itemize}

    \item \textbf{Frequency Response (Time Permitting)}
    \begin{itemize}
    \item Bode plots
    \item Gain margin and phase Margin
    \end{itemize}
\end{itemize}

This is a junior level core course, and as such, you will be expected to perform a large amount of work. The subject of system dynamics and analysis requires a good knowledge of concepts covered in your dynamics class, differential equation class, linear algebra and computing. All these factors are going to make this course more difficult, \textbf{so be prepared to invest a lot of time in this course.} We promise to help you in every manner possible to learn the material being covered in the class and to achieve your professional goal provided you are prepared to invest a quality time in this course. \textbf{The payoff will be not only a deeper understanding of the subject but also an improved ability to think critically and independently.}

You are \textbf{personally responsible} for all information disseminated during the lectures. This means knowing all assignment due dates and knowing all material, handouts, and announcements made in the lectures, whether or not you were present. Thus, if you miss a lecture, it is your responsibility to obtain all information presented during that lecture.

\subsection*{Assignments \& Grading}
The final course grade will be calculated based on the following percentages. The exam
with your highest score will be weighted with an \textbf{additional 2\%.}
\begin{table}[h!]
    \centering
    \renewcommand{\arraystretch}{1.2}
    \begin{tabular}{p{4cm}|p{2cm}|p{2cm}|p{6.5cm}}
        \hline
        \textbf{Item} & \textbf{Weight} & \textbf{Weight*} & \textbf{Frequency}  \\\hline
        \textbf{Short Quizzes: }& 10\% & 10\% & Every week\\
        \textbf{Homework Quizzes: }& 20\% & 20\%& $\sim$ 5-6 in the semester\\
        \textbf{Projects:} & 20\% & 7\%& 3 in the semester \\
        & & & (1 for Funduino option) \\
        \textbf{Mid-Term 1: }& 16\% & 16\% & Feb 13, 6:15 pm - 7:30 pm \\
        & & & Location: TBD\\
        \textbf{Mid-Term 2:} & 16\% & 16\%& Mar 31, 6:15 pm - 7:30 pm \\
        & & & Location: TBD\\
        \textbf{Final Exam: } & 16\% & 16\% & May 7, 6:50 pm - 8:40 pm \\
        & & & Location: TBD\\
        \textbf{Funduino*:} & 0\% & 15\%  & \\
        \hline
    \end{tabular}
\end{table}
The final letter grades are assigned based on your performance relative to the overall class performance. The letter grade boundaries are assigned to make sure that the class average is close to B or B- boundary.

\subsubsection*{Short Quizzes:} Several short quizzes (10 mins max duration) will be conducted throughout the semester to solidify your understanding of the material covered in previous classes and promote discussion on class material. On average one short quiz will be conducted every week though there can be multiple quizzes held in a week or no quiz held in some weeks. These quizzes will be conducted through CANVAS. \textbf{At least one lowest quiz scores from the semester will be dropped.}

\subsubsection*{Homework Quizzes:} Several Homework (HW) assignments will be assigned throughout the semester to cover basic concepts taught in the class. \textbf{However, assigned HW problems will not be collected or graded.} Students will be expected to understand all problems in each problem set, and a quiz (20-30 minutes) will be conducted in the class on the due date of the homework assignment. \textbf{The lowest HW quiz score from the semester will be dropped.}

Assignment solutions will be posted on CANVAS and/or discussed in class after the quiz. Quiz problems will be taken directly from the assigned homework problems. Hence, it is to your advantage to understand all of the homework problems. System dynamics and analysis is not a subject that can be learned simply by reading the book or class notes. Working problems is the only way to solidify your understanding of the material. You are encouraged to get help from instructor, teaching assistants or other students in the course for problems that are particularly difficult.
\begin{table}[h]
\begin{center}
    \begin{tabular}{|c|c|}
        \hline
        \textbf{HW Quiz} & \textbf{Date} \\ \hline
        1 & date \\ \hline
        2 & date \\ \hline
        3 & date \\ \hline
        4 & date \\ \hline
        5 & date \\ \hline
        6 & date \\ \hline
        7 & date \\ \hline
    \end{tabular}
\end{center}
\end{table}

\subsubsection*{Projects:} Three projects will be assigned throughout the semester, which will involve both analytical and numerical work. You will be required to write several computer programs in your favorite programming language as a part of your projects. \textbf{These projects will require group effort and at max three students will be allowed in a group.} Each group will submit one project report/solution, which should include problem formulation, solution methodology, results and discussion. Copying of solutions or computer programs will be considered a violation of the University honor code. There will be penalties for late project report submissions: \textbf{30\% deduction each day. } \textbf{Special exceptions} will be considered on an individual basis and only if the instructor has been contacted at least a day before the due date of the project.

\textit{\textbf{Funduino Project:}} Funduino project is optional and will involve extensive coding while making use of Arduino board to learn engineering application of material being covered in the class. It is usually split in two parts (two reports and demonstrations). Interested students will organize themselves in a group of three maximum and will be assigned a group grade. Throughout the semester, student groups will work on the assigned Arduino project with the help of the instructor \& teaching assistants. They will be required to provide demos throughout the semester and submit a mid-term report and a final report at the end of the semester. Students opting for this option will be required to buy appropriate hardware by their own. \textbf{ You will need to opt for this option by the end of third week of classes.} Students opting for Funduino option will be \textbf{required to do only the first Matlab project.} Note that student opting for this option \textbf{can earn $2\%$ extra points.} The Funduino project will account for $15\%$ of your final grade and the other $7\%$ will come from the first Matlab project.

\subsubsection*{Exams:} There will be two mid-term exams and one comprehensive final exam. Exam problems may be taken directly from quizzes and the homework problems with some modifications. Thus, in addition to \textbf{the weight placed on homework in the final grade, it is to your advantage to understand all of the homework problems in the textbook.} All exams will be closed-book and closed notes and no calculator \& phones will be allowed. You will be allowed to bring one paper sheet with key formulas written on it but \textbf{no problem solutions}, and you will submit this paper sheet with your exam.

If you do not agree with the grading of a particular exam or quiz problem, you will have one week from the date the exam/quiz is returned to submit a written explanation of why you think the grade should be higher. However, the final decision will remain the instructor's.

\textit{Make-up exams:} Make-ups will be given for legitimate reasons such as illness, personal or family emergency, religious observance, Penn State athletic commitment, job interview, and et cetera. Note that sleeping late or forgetting to take the exam are not acceptable reasons. Please let us know ahead of time (at least two days before the scheduled exam) if you will need to take a make-up, but in any event, contact us as soon as possible to make the arrangements. \textbf{No make-ups will be provided for HW Quizzes.}

\subsection*{Academic Integrity}
Penn State and the College of Information Sciences and Technology are committed to maintaining \href{https://undergrad.psu.edu/academic-information/academic-integrity.html}{Penn State's policy on Academic Integrity} in this and all other courses. We take academic integrity matters seriously and expect you to become a partner to the University/College standards of academic excellence.

For more information, please review these policies and procedures: \href{https://student.worldcampus.psu.edu/a-z-index/academic-integrity}{Penn State World Campus Academic Integrity Resources}

While utilizing additional sources outside of this class is encouraged for gaining a better understanding of course concepts, seeking explicit answers for graded assignments from outside sources (e.g. Course Hero, Chegg, tutoring services like tutor.com, etc.) is considered CHEATING and will not be tolerated.  Sanctions range from failure of the assignment or course to dismissal from the University. Additionally, sharing course content without permission is a violation of copyright and may result in university sanctions and/or legal ramifications. Contact your instructor with questions related to this topic.

\subsection*{Use of Generative AI}
According to Penn State policy (\href{https://undergrad.psu.edu/aappm/G-9-academic-integrity.html}{G-9 Academic Integrity}), an academic integrity violation is “an intentional, unintentional, or attempted violation of course or assessment policies to gain an academic advantage or to advantage or disadvantage another student academically.” Unless your instructor tells you otherwise, you must complete all course work entirely on your own, using only sources that have been permitted by your instructor, and you may not assist other students with papers, quizzes, exams, or other assessments. If your instructor allows you to use ideas, images, or word phrases created by another person (e.g., from Course Hero or Chegg) or by generative technology, such as ChatGPT, you must identify their source. You may not submit false or fabricated information, use the same academic work for credit in multiple courses, or share instructional content. Students with questions about academic integrity should ask their instructor before submitting work.

Students facing allegations of academic misconduct may not drop/withdraw from the affected course unless they are cleared of wrongdoing (see \href{https://undergrad.psu.edu/aappm/G-9-academic-integrity.html}{G-9 Academic Integrity}). Attempted drops will be prevented or reversed, and students will be expected to complete course work and meet course deadlines. Students who are found responsible for academic integrity violations face academic outcomes, which can be severe, and put themselves at jeopardy for other outcomes which may include ineligibility for Dean’s List, pass/fail elections, and grade forgiveness. Students may also face consequences from their home/major program and/or The Schreyer Honors College.

\subsection*{University Policies}
Review current information regarding various Penn State policies (such as copyright, counseling, psychological services, disability and military accommodations, discrimination, harassment, emergencies, trade names, etc.) on the \href{https://docs.google.com/document/d/1FIQdII2qw3SJOIgQWTWRByCxSbsnY6DcZA0JHzL4yBk/pub}{University Policies page}.

Penn State takes great pride to foster a diverse and inclusive environment for students, faculty, and staff. Acts of intolerance, discrimination, or harassment due to age, ancestry, color, disability, gender, gender identity, national origin, race, religious belief, sexual orientation, or veteran status are not tolerated and can be reported through Educational Equity via the Report Bias webpage. (http://equity.psu.edu/reportbias/).

\subsection*{Military Accommodations}
Veterans and/or currently-serving military personnel and/or spouses with unique circumstances (e.g., upcoming deployments, drill/duty requirements, disabilities, VA appointments, etc.) are welcomed and encouraged to communicate these to the instructor - in advance, if possible - in the case that special arrangements need to be made. Please review these Services for Military and Veteran Students.

\subsection*{Disability Accommodation}
Penn State welcomes students with disabilities into the University’s educational programs. Every Penn State campus has an office for students with disabilities. Student Disability Resources (SDR) website provides contact information for every Penn State campus (http://equity.psu.edu/sdr/disability-coordinator). For further information, please visit Student Disability Resources website (http://equity.psu.edu/sdr/).

In order to receive consideration for reasonable accommodations, you must contact the appropriate disability services office at the campus where you are officially enrolled, participate in an intake interview, and provide documentation: See documentation guidelines (http://equity.psu.edu/sdr/guidelines). If the documentation supports your request for reasonable accommodations, your campus disability services office will provide you with an accommodation letter. Please share this letter with your instructors and discuss the accommodations with them as early as possible. You must follow this process for every semester that you request accommodations.

\subsection*{Counseling and Psychological Services}
Many students at Penn State face personal challenges or have psychological needs that may interfere with their academic progress, social development, or emotional wellbeing. The university offers a variety of confidential services to help you through difficult times, including individual and group counseling, crisis intervention, consultations, online chats, and mental health screenings. These services are provided by staff who welcome all students and embrace a philosophy respectful of clients’ cultural and religious backgrounds, and sensitive to differences in race, ability, gender identity and sexual orientation.\\
Counseling and Psychological Services at University Park (CAPS): http://studentaffairs.psu.edu/counseling/, Contact: 814-863-0395\\
Counseling and Psychological Services at Commonwealth Campuses (https://senate.psu.edu/faculty/counseling-services-at-commonwealth-campuses/)\\
Penn State Crisis Line (24 hours/7 days/week): 877-229-6400\\
Crisis Text Line (24 hours/7 days/week): Text LIONS to 741741

\subsection*{Educational Equity and Reporting Bias}
Penn State takes great pride to foster a diverse and inclusive environment for students, faculty, and staff. Acts of intolerance, discrimination, or harassment due to age, ancestry, color, disability, gender, gender identity, national origin, race, religious belief, sexual orientation, or veteran status are not tolerated and can be reported through Educational Equity via the Report Bias webpage (http://equity.psu.edu/reportbias/).

% End here
\end{document}